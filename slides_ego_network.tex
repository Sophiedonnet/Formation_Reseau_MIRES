\documentclass{beamer}
% , serif, professionalfont
%\documentclass{beamer}
\usepackage{pxfonts}
\usepackage{eulervm}

\usepackage{amsthm,amsmath,amsbsy,amssymb,amsfonts,pstricks,fancybox,bbm,pxfonts,dsfont,marvosym,pifont,fourier}

\usepackage{pstricks}

\usepackage{comment}

\usepackage[utf8]{inputenc}
\usepackage[T1]{fontenc} 
\usepackage[francais]{babel}

\usepackage[]{algorithm2e}
%\mode<article> % only for the article version
%{
%  \usepackage{beamerbasearticle}
%  \usepackage{fullpage}
%  \usepackage{hyperref}
%}


%\usepackage{pstricks}
%\usepackage{tikz}
%\usetikzlibrary{plothandlers,plotmarks,arrows,automata}
%\usepackage{amsmath,amssymb}

%\usepackage{colortbl}
%\usepackage{mathrsfs}
%\renewcommand{\mathcal}{\mathscr}
%\usepackage{enumerate}
%\usepackage{lmodern}


\newcommand{\disp}{\displaystyle}
\newcommand{\ds}{\displaystyle}
\newcommand{\vs}{\bigskip}
\newcommand{\esp}{\mathbb E}
\newcommand{\prob}{\mathbb P}
\newcommand{\ee}{\mathrm e}
\newcommand{\ii}{\mathrm i}
\newcommand{\dd}{\text{d}}
\newcommand{\Int}{\mathrm{int}\,}
\newcommand{\Cl}{\mathrm{adh}\,}
\newcommand{\Supp}{\operatorname{Supp}}
\newcommand{\Card}{\operatorname{Card}}
\newcommand{\Var}{\operatorname{Var}}
\newcommand{\Ave}{\operatorname{Ave}}
%\renewcommand{\mathcal}{\mathscr}

\newcommand{\logit}{\operatorname{logit}}
\renewcommand{\epsilon}{\varepsilon}
\newcommand{\eps}{\varepsilon}
\newfont{\manfnt}{manfnt}
%\newcommand{\danger}{{\manfnt\symbol{'177}}}
\newcommand{\mdanger}{\marginpar[\hfill\danger]{\danger\hfill}}
\newcommand{\new}{{\manfnt\symbol{30}}}
\newcommand{\mnew}{\marginpar[\hfill\new]{\hfill\new}}
\renewcommand{\P}{\mathbb{P}}
\newcommand{\tbrown}[1]{\textcolor{brown}{#1}}
\DeclareTextFontCommand{\hershey}{\fontfamily{hscs}\selectfont}
\newcommand{\tb}{\color{brown}}

%\usetheme{Frankfurt}
%\usetheme{Boadilla}
\usetheme{Boadilla}%Pittsburgh}
%\usecolortheme{seahorse}
\usecolortheme{rose}


\newcommand{\R}{\mathop{\mathbb{R}}}
\newcommand{\N}{\mathop{\mathbb{N}}}
\newcommand{\EE}{\mathbb{E}}
\newcommand{\PP}{\mathbb{P}}
\newcommand{\LL}{\mathbb{L}}
\newcommand{\F}{\mathcal{F}}
\def\argmin{\mathrm{argmin}}
\def\argmax{\mathrm{argmax}}
\def\1{1}
\definecolor{lightblue}{RGB}{223, 242, 255}
\definecolor{lightred}{RGB}{254,191,210}



\makeatletter
\newcommand{\setnextsection}[1]{%
  \setcounter{section}{\numexpr#1-1\relax}%
  \beamer@tocsectionnumber=\numexpr#1-1\relax\space}
\makeatother


\setbeamertemplate{itemize items}[triangle]
\setbeamercolor{itemize item}{fg=brown}
\setbeamertemplate{enumerate items}[circle]
\setbeamercolor{item projected}{bg=brown}
\setbeamercolor{block title}{fg=brown!100, bg=gray!20}
\setbeamercolor{block body}{bg=gray!10}
\graphicspath{{fig/}}

\title[Ego-Network]{Analyse de réseaux égo-centrés}
\author[Nicolas Verzelen]{Nicolas~Verzelen}
% \institute{  Université Montpellier 2\\~\\
% \onslide<2>{
% \includegraphics[width=3cm]{marin.jpg} ~~
% \includegraphics[width=2.7cm]{xian.jpg} ~~
% \includegraphics[width=1.5cm]{cornuet.jpg}
% }
% }
\institute[INRA]{INRA}
\date[Formation Analyse de réseaux]{}
%\subject{ABC in London}

\AtBeginSection[]{%
  \begin{frame}<beamer>
    \frametitle{Plan}
    \tableofcontents
  \end{frame}
}

\begin{document}











\section{Ego-réseau et une analyse ego-centrée?}



\begin{frame}
 \frametitle{D'un réseau à des réseaux ego-centrés}
 
 {\bf Définition}: Le réseau ego-centré d'un noeud est le réseau formé par un noeud et ses voisins. 
 
 \begin{center}
 \includegraphics[width= 5cm]{reseau1.png} 
 
 \includegraphics[width= 1.5cm]{egonetwork_1.png} 
  
 \end{center}

 
 
\end{frame}


\begin{frame}
 \frametitle{Qu'est-ce qu'une analyse ego-centrée?}
 
 

Dans des analyses ego-centrées, on s'intéresse typiquement à des indicateurs/ mesures sur les noeuds (ex: degrés) . 

 
 

 \medskip 
 
 En calculant des indicateurs sur les noeuds, on peut ensuite les incorporer dans des méthodes d'analyse statistiques plus classiques.
 
 
 {\bf Ex: } Pour expliquer le nombre d'espèces cultivées dans une ferme, ajouter à des variables plus classique (ex: taille de la ferme) le degré de la ferme dans le réseau d'échanges. 
 
 
 
 \bigskip 
 
 
 Les analyses égo-centrées sont surtout intéressantes si les questions d'intérêt portent sur les noeuds. 
 
 
\end{frame}








\begin{frame}
\frametitle{Cet exposé }


\begin{itemize}
 \item Quelles analyses lorsque le réseau  {\color{red} faiblement échantillonné}  de type  {\color{red} ego-centré} ou {\color{red} snow-ball}? 
 \item Procédures d'analyse {\color{red} Ego-centrées}
\end{itemize}


 
\end{frame}


















\section{Echantillonnage et ego-réseaux}



\begin{frame}
\frametitle{Echantillonnage de réseaux} 
 
 Le Réseau est rarement  complétement observé.
 
 
 \begin{center}
 \includegraphics[width= 5cm]{reseau1.png}  
 \end{center}

 
 
 On s'intéresse ici à des données pour lesquelles l'effort d'échantillonnage est faible.
\end{frame}




\begin{frame}
 \frametitle{Echantillonnage d'ego-réseaux}
 
 {\bf Observations}: Collection d'ego-réseaux 
 
 \begin{center}
 \includegraphics[width= 5cm]{egonetwork.png}  
 \end{center}

 
 
 Deux types de noeuds
 \begin{itemize}
  \item Noeud explorés
  \item Noeud découverts mais non explorés (en rouge)
 \end{itemize}

{\bf Difficulté}: Si l'effort d'échantillonnage est faible (au regard de la taille du réseau), 
 
\end{frame}


\begin{frame}
\frametitle{Echantillonnage en boule de neige}

{\bf Objectif}: Mieux découvrir localement le réseau


 \begin{center}
 \begin{figure}
 \includegraphics[width= 4.5cm]{egonetwork.png}\hspace{1cm }\includegraphics[width= 5cm]{snowball2.png}   \caption{Etapes 1 et 2 de la boule de neige} 
 \end{figure}

 
 

 \end{center}

{\bf Principe}: Echantillonnage d'ego-réseaux puis  exploration de certain noeuds découverts à la première étape,\ldots, 








\end{frame}





\section{Analyse globale pour des ego-réseaux (ou des boules de neige)}


\begin{frame}
\frametitle{Objectifs}

\begin{itemize}
 \item Quelle caractéristique globale peut-on estimer avec des ego-reseaux? 
 \item Quelle caractéristique globale peut-on estimer avec un échantillonnage boule de neige?  
\end{itemize}


\vspace{1cm}

{\bf Package utilisé}: {\it igraph}
 
\end{frame}




\begin{frame}
 \frametitle{Distribution des degrés}
 
 {\bf Caractéristiques}: 
 \begin{itemize}
  \item Toujours possible à estimer.
  \item Attention à ne pas compter les degrés des noeuds non visités!
  \item (biais vers les hauts degrés pour l'échantillonnage boule de neige)
 \end{itemize}

 
 
 
 
\end{frame}




\begin{frame}
 \frametitle{Homophilie}
 
 {\it Homophilie ou assortativité}: Propension d'individus semblables (ex: âges, liens de parenté,...) à être connecté.
 
 
 \medskip 
 
 {\bf Caractéristiques}: 
 \begin{itemize}
  \item Nécessite de connaître les caractéristiques (ex: âge, liens de parenté) des individus non visités. 
  \item Sinon ... uniquement estimable à partir des couples d'individus visités. 
 \end{itemize}

 
\end{frame}

\begin{frame}
 \frametitle{Reciprocité}
 
 Pour un réseaux dirigé, propension de deux noeuds à avoir deux liens réciproque.   
 
 
 \medskip 
 
 {\bf Caractéristiques}: 
 \begin{itemize}
  \item Nécessite de connaître les arêtes partant et arrivant des noeuds visités. 
  \item Sinon ... uniquement estimable à partir des couples d'individus visités. 
 \end{itemize}

 

 
\end{frame}


\begin{frame}
 \frametitle{Transitivité}
 
 Propension de deux noeud reliés à un même noeud à être relié entre eux. 
 
 
 
 
 \bigskip 
 
 {\bf Caractéristiques}: 
 \begin{itemize}
  \item Nécessite au moins un échantillonnage boule de neige à 2 étapes.
 \end{itemize}
 
 
 
\end{frame}






\section{Analyse ego-centrée}


\begin{frame}
 
 
 {\bf Principe général}: 
 \begin{enumerate}
  \item Calculer des indicateurs de la position du noeuds dans le   du réseau
  \item Comparer ces indicateurs à d'autres caractéristiques du noeuds
 \end{enumerate}

 
 
 
 
\end{frame}

\subsection{Indicateurs de la position du noeud dans le réseau}




\begin{frame}



\frametitle{Différents indicateur d'importance/centralité des noeuds}



{\it Sensible =  nécessite un effort d'échantillonnage important!} 


\medskip 

{\bf Mesures de centralité}
\begin{enumerate}
 \item Degree [Peu sensible]
 \item Eigen-centrality [Sensible!]
 \item Closeness [Sensible!]
 \item Betweeness [Très sensible!!]
\end{enumerate}




\end{frame}







\subsection{Relier la position du noeud et caractéristiques du noeud }




\begin{frame}
 \frametitle{Lier centralité et caractéristiques du noeud}
 
 la comparer à d'autres caractéristique du noeud (âge, richesse), en utilisant des méthodes statistiques multivariées classique: 
 \begin{itemize}
  \item Régression linéaire, régression logistique 
  \item tests du $\chi^2$. 
  \item Analyse en composantes principales, ...
 \end{itemize}

 
 
\end{frame}






\subsection{Prédiction de lien}

\begin{frame}
\frametitle{Prédiction de lien}

Il ne s'agit pas à proprement d'une 'ego-analyse'. 

\bigskip

{\bf Objectif}: Essayer de prédire/expliquer la présence de lien à partir de caractéristique des noeuds (ex: Age des protagonistes)  ou du couple de noeud (ex: lien de parenté)


\medskip 


{\bf Outils}: Régression logistique (ou linéaire)





\end{frame}



\begin{frame}
\frametitle{Bibliographie}


\begin{itemize}
 \item Eric D. Kolaczyk, Gábor Csárdi (auth.)- {\bf Statistical Analysis of Network Data with R}-Springer-Verlag New York (2014)
 \item 
\end{itemize}



\end{frame}







\end{document}























